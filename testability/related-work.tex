\section{Related Work}\label{sec:relatedwork}

There has been little to no research in the literature regarding testing canvas applications. Due to the absence of an object model for canvas elements, it is difficult to apply conventional web testing techniques to canvas elements.

Nonetheless, there exist a few open-source attempts, which can be used to perform  basic testing of canvas elements. However, these open-source tools are not part of a research publication, and therefore they lack detailed experimentation or a thorough explanation of the methodology. We briefly discuss some of these tools. 

Canteen\footnote{https://github.com/platfora/Canteen} takes a callstack analysis approach. The tool captures a stack of the function calls sent to the canvas element, then compares the runtime stack against a known correct call stack manually provided by the developer. A major disadvantage of this approach is that it focuses exclusively on the call stack, meaning a test's pass or failure does not directly correspond to the visual state of the canvas. That is, the actual rendered canvas that the end user observes is not tested.

Other open-source tools, such as Needle\footnote{https://github.com/bfirsh/needle} and JS-ImageDiff\footnote{https://github.com/HumbleSoftware/js-imagediff}, adopt a visual approach instead of code analysis. Runtime screenshots are compared against known good screenshots provided by the test writer. However, being a direct image differencing approach, it shares much of the fragility issues~\cite{coppola_automated_2016, leotta_visual_2014} of visual approaches, where even a single pixel could result in test failure. Furthermore, this approach still requires test writers to provide \textit{a priori} visual oracles.

There is another related body of literature that is related to visual-based approaches for testing web pages in general, but not for canvas testing. For example, WebDiff~\cite{choudhary2010webdiff} detects cross-browser incompatibilities for a given webpage. It performs an indirect comparison of the appearance of the webpage in two browsers using DOM-based analysis. Although the tool subsequently confirms the initial DOM-analysis using a visual comparison, the approach still requires the DOM to detect the cross browser incompatibility in the first place, and uses a visual comparison as a confirmation. Although the tool was shown to be helpful in terms of general web page cross-browser testing, it cannot be used with canvas elements because they lack a DOM representation. 

Another approach, using the WebSee~\cite{mahajan16apsec} tool, targets the problem of detecting and locating visual inconsistencies in web applications. The approach uses visual comparison to detect visual differences between pages, and then locates the inconsistency using DOM elements. While the approach showed good performance in detection and localization of inconsistencies, it does require the DOM in its operation and therefore can not be used with canvas elements.

PESTO~\cite{leotta2015automated} aims at simplifying the creation of visual web tests. The approach starts with a given DOM-based test suite and then generates visual locators. It then concludes by generating a visual test suite that matches original DOM test suite. While it shows good performance, canvas elements can not be used with this tool because they do not have a DOM.

Scry~\cite{burg2015explaining} focuses at assisting developers in explaining and reproducing visual changes on a web page. It asks the developer to specify which webpage element to monitor, and then watches that element. Whenever the appearance of the element changes, the method stores the DOM and CSS of that element in order to determine the code changes that produced the appearance change. While this tool has interesting applications and could help explain appearance changes in a web page, it can not be used for canvas elements because it is based on analyzing the DOM of elements. Canvas elements, however, have no DOM-tree representation, which makes it incompatible with such DOM-based tools.

