\section{Conclusions}
\label{sec:conclusions}

Web applications based on canvas elements allow the creation of dynamic graphics, interactive user interfaces, and scalable visualizations. However, there has been little to no research in literature in terms of testing canvas elements. This chapter proposed a testing approach, implemented in a tool, \tool, based on visual analysis of the screenshot of canvas elements, and generating an augmented DOM tree for the canvas element to allow making test assertions on it. We evaluated the accuracy of the proposed approach and its effectiveness in detecting faults injected in canvas elements. We found the inference process to be relatively accurate (around $91\%$ accuracy on average) with a true positive rate of $93\%$ in detecting fault injections. 
We note, however, that the goal of this work is to provide developers with the fundamental capability of observing the canvas state and making assertions on it. However, it does not provide a complete testing solution, but rather make it \emph{possible} to perform the testing process itself, thereby improving testability. As part of future work, a more through canvas testing solution can be provided such that it will cover more complex and resizable canvas elements, and generate the tests in a fashion that developers might prefer (e.g., using relative positioning in assertions). 