\section{Introduction}\label{sec:introduction}
Canvas elements are one of the major web technologies used to deliver interactive and dynamic graphics-intensive web applications. Canvas-based web applications are utilized in a wide variety of fields, such as math and science education~\cite{arreola2017creating}, geographical modelling~\cite{christen2012web}, and genetics research~\cite{buels2016jbrowse}. 
The \texttt{<canvas>} element on the HTML page is only a container for graphics, which is updated programmatically, via its APIs, through JavaScript code.

From a testing perspective, however, there has been little to no progress in the literature in terms of testing canvas elements. None of the works surveyed in Chapter 2 perform canvas testing. In addition, existing web testing methodologies do not apply to canvas elements for the following reasons. The testing of web applications in practice is based on using a browser automation tool, such as Selenium\footnote{http://www.seleniumhq.org/} or PhantomJS\footnote{http://www.phantomjs.org/}, to make assertions on the DOM (Document Object Model) of the webpage, which is a tree representation of the current set of nodes in a page. This approach opens the web application and analyzes the dynamic DOM tree, and allows developers to write tests that interact with DOM elements and assert their various attributes. However, these DOM-based approaches cannot be used with canvas elements.  These elements only expose a low-level graphics API, which allows directly painting on the screen pixel-by-pixel, reducing the browser overhead and improving the final real-time speed of the web application.
As such, the canvas element does not have a representative DOM-tree for its internal structure and properties that can be tested directly using existing web testing tools and techniques. 
 
An alternative approach of testing web applications is visual testing, using tools such as  Sikuli\footnote{http://www.sikuli.org/} and eggPlant\footnote{https://www.testplant.com/}. Visual testing relies exclusively on the application's appearance, with no access to the application's DOM tree. All these methods rely on a visual comparison between a number of initial screenshots (i.e., visual locators) created \textit{a priori} by the tester, and comparing them to screenshots at runtime during test execution.
Existing visual approaches have several limitations with respect to testing canvas elements, namely, (1)~the set of locators (i.e., screenshot images) need to be initially gathered before starting any testing activity. This can be difficult and time consuming, given that each canvas element may be associated with many different screenshots, depending on its dynamic states; (2)~the creation of assertions (e.g., asserting objects properties or locations) is not fully supported in these tools. More specifically, a visual test consists of a image comparison, which does not provide information regarding the objects on the canvas, such as their shape, location, or color, etc; 
(3)~from a maintenance viewpoint, the commonly used visual testing approach of image diffing has been shown to be more fragile compared to DOM-based methods~\cite{coppola_automated_2016,leotta_visual_2014}, because changing even a slight change in one pixel in the image would cause a false positive (i.e., false test failure). Therefore, a large number of visual tests needs to be rewritten for every application version because of the fragile nature of visual analysis used in these approaches.

In this chapter, we propose a novel approach for testing canvas elements that combines aspects of both visual and DOM testing. The approach begins with a visual analysis of the canvas screenshot, infers objects in terms of their shapes, properties, and relationships, and then generates an augmented DOM for the canvas element representing the visual contents of the canvas. This makes canvas elements testable using widely used DOM-based testing techniques,  without requiring \textit{a priori} visual locators. In addition, our approach automatically generates tests to check the inferred objects and their properties on the canvas. We implement this approach in a tool called \tool, and evaluate its accuracy and fault detection performance.
Our work makes the following main contributions: 

\begin{enumerate}
\item The first approach for making web canvas elements testable through visual analysis, to the best of our knowledge.
\item A technique for inferring objects, their shapes, properties, and relationships from images of canvas elements and representing them as elements in an augmented DOM.
\item An implementation of our approach, called \tool, which supports testing web applications that are entirely canvas-based, as well as web applications that contain only a few canvas elements.
\item An empirical evaluation on 50 canvas screenshots from five canvas web applications. Our results show that \tool has an average accuracy of $91\%$, and it can detect $93\%$ of injected faults successfully.

\end{enumerate}
