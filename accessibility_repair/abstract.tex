\begin{abstract}

Filling web forms is a common online activity. 
Web forms are made accessible to users with disabilities 
by conveying their content through specific DOM labeling 
markups. The absence of these markups is one of the most 
common accessibility errors. However, there is currently 
little to no work in terms of having an automated analysis 
process that allows inferring the labeling markups in order 
to automatically make forms accessible for users with disabilities. 
In this paper, we propose a web form analysis approach that 
infers labels by first constructing different types of 
visual cues from a form, then optimizing the combination of 
various cues and form fields, and finally augmenting the DOM 
to incorporate the required labeling markup. We evaluate 
our approach on real-world subjects and assess the accuracy 
of labeling inference, the safety of the DOM augmentation, 
as well as the labeling performance. The results show an 
average F1-measure of 88.4\% for label inference, and an 
average run-time of around 1.6 seconds.

\end{abstract}

\keywords{web forms, accessibility errors, 
accessibility repair, visual analysis}
