\section{Conclusion}

Filling web forms is a key activity when browsing the web. 
While this task is routine for sighted users, 
it presents a significant hurdle for non-sighted users if the form does not 
contain the required DOM accessibility labeling markup. This 
issue of missing labeling markup is one of the most common accessibility 
errors. However, 
when a non-sighted user is faced with missing form labeling, 
there are currently little to no options available to access 
that form. To this end, this chapter proposed a software analysis approach 
that automatically analyzes web forms and infers their labels to make them accessible. 
The approach first abstracts a given web form, then generates 
visual cues from the form. These cues are then used in an optimization 
model to solve for the form labeling associations. These are 
finally translated into standard ARIA accessibility markups and augmented into 
the DOM to repair the form and make it accessible. 
We evaluated our approach on 30 real-world subjects 
and assessed the accuracy of labeling inference, the safety of 
the DOM augmentation repairs, as well as the labeling performance. 
The results show an average F1-measure of 88.4\% for label 
inference, and an average run-time of around 1.6 seconds.
