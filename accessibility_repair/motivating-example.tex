% !TEX root =  paper.tex

\section{Background and Motivating Example} \label{sec:motivation}
\Cref{fig:motivating-example} shows an example of
an inaccessible web form. 
By looking at the rendered form in \Cref{fig:motivating-example}b,
a sighted user can understand 
the structure and content of the web form and navigate their way through it.
For instance, the user would recognize 
that they can write down their name, email, and indicate if they 
would like to receive daily emails. Most importantly, the user recognizes 
which form fields are associated with the name or the email, for instance. 
The collection of information we just described is referred to as form labeling. 
It indicates what fields are present on the form and specifies the textual 
label that describes what each field represents. 

While this understanding of form content and labeling happens naturally when 
sighted users look at the form, that is not the case for visually impaired 
(i.e., non-sighted) users due to the absence of visual perception. 
In \emph{inaccessible} forms, such as the one in \Cref{fig:motivating-example}, 
the labeling is communicated exclusively through visual design. 
This is because the HTML markup in \Cref{fig:motivating-example}a 
is just a collection of \code{<div>}s, \code{<p>}s, and \code{<input>}s 
that do not communicate any labeling associations to indicate what the various 
elements represent. For instance, in \Cref{fig:motivating-example}a, there is no 
piece of markup indicating what the \code{<textarea>} represents or means. 
That is, there is no markup that describes to a non-sighted user 
what information they are supposed to provide for that form field. 
The form is therefore unusable in this case. 
In contrast, sighted users understand 
that the \code{<textarea>} is where they should write down their message, 
or that the first text input is where they should provide their name. 
They understood this from the visual design and layout 
of the form. This implicit visual communication is natural for sighted 
developers and users, but is unavailable for users who can 
not have access to visual information due to disabilities. 
Accordingly, the web form is deemed inaccessible since its markup does 
not specify the labels of the form elements. 


%\subsection{Accessible Rich Internet Applications}\label{subsec:aria-roles}
%Non-sighted users rely on screen readers  
%to parse a web page for them since they can not directly 
%perceive the page. 
%Screen readers are tools that speak out the various information, 
%forms, and labels present on the page, and the non-sighted 
%user would then select one of the 
%options they heard. 
%While a standard web browser simply renders the page as is and leaves 
%it to the end user (i.e., a sighted user) to visually understand what the 
%various elements mean or represent, 
%screen readers require that the page markup explicitly 
%indicates any semantic information (e.g., form labeling)  
%in order to determine what information to convey audibly to the users. 
%
%The standard markup that is used by screen readers during their processing 
%of pages is the World Wide Web Consortium's (W3C) \emph{Accessible Rich 
%Internet Applications} (ARIA)~\cite{ARIA} standard. 
%The ARIA standard specifies a set of markup attributes that 
%should be included in the page's HTML to make it accessible to screen readers. 
%Accordingly, when web developers incorporate these markups into their pages, 
%the result is an accessible app that non-sighted users can use through their 
%screen readers. 
%
%\subsection{Syntax checks}\label{subsec:syncheckers}
%Existing accessibility testing tools~\cite{yesilada2019web,ukgov:audit:2018} 
%are based on syntactic checking. 
%They check the HTML against certain syntax rules. 
%For instance, one common check is to assert that every \code{img} element 
%has an \code{alt} text attribute for textual image descriptions. 
%Another check asserts that \code{input} elements must not 
%be descendant of \code{a}.
%Another example is checking that every \code{ul} list 
%element has a non-empty list item \code{li} child.  
%In general, all syntax checks follow the same template: 
%if certain syntax A is present, 
%assert that syntax B is true. 
%%Established accessibility guidelines (e.g., Web Content Accessibility Guidelines -- WCAG~\cite{WCAG}) 
%%includes a number of such syntax rules as well as other guidelines. 
% 
%While syntax checks can be useful and simple 
%assertions, and are easy to automate, 
%syntax checks are not capable of addressing the more important, 
%and more challenging, analysis that is 
%required to infer and repair form labeling. 
%To illustrate this, we revisit \Cref{fig:motivating-example}(b), 
%where we can visually see that there is a text box form field 
%whose label is `Message`. 
%We perceived this labeling association by visually looking at 
%the rendered form in (b).  
%However, when we look at the markup in (a) we do not find that 
%labeling expressed in the markup, 
%and therefore conclude that the form is inaccessible.  
%We now pause to reflect and observe that there is no possible 
%syntactic check 
%that can automate the conclusion we just reached. 
%Checking that an attribute is missing is simple and readily 
%automatable, but determining 
%what values it should contain is difficult. 
%The steps that we just walked through requires 
%our own visual perception as sighted developers or users in 
%order to reach the conclusion 
%that there is a mismatch between the visually perceived labeling 
%of the from, and the form's markup. 
%
