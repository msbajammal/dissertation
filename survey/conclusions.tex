% !TEX root =  manuscript.tex
\section{Conclusions}\label{sec:conclusions}
A recent and growing trend in software engineering 
research is to adopt a \emph{visual perspective} of 
the software, which entails extracting and processing 
\textit{visual artifacts} relevant to the software being analyzed. 
To gain a better understanding of this trend,
in this chapter, we surveyed the literature on the use of 
visual analysis approaches in software engineering. 
From more than \initialPoolSize publications, 
we systematically obtained \numberOfPapers papers 
and analyzed them according to a number of research dimensions.
Our study revealed that visual analysis techniques 
have been utilized in all areas of software engineering, 
albeit more prevalently in the software testing field.
We also discussed why visual analysis was utilized, 
how these techniques are evaluated, and what limitations they bear.
Our suggestions for future work include the development 
of common frameworks and visual benchmarks to collect 
and evaluate the state-of-the-art techniques, 
to avoid relying on ad-hoc solutions. We believe that 
the findings of this work illustrate the potential of 
visual approaches in software engineering, and may help 
newcomers to the field in better understanding the research landscape.

\hl{A number of key findings can be observed from the survey in order to help in 
guiding and framing the remainder of the dissertation. 
First, cross-browser testing is by far the most common area of application, 
whereas exploring non-functional properties received little to no focus. 
This shows that there is an opportunity to explore the use of visual analysis 
in improving non-functional properties, and therefore the remainder of the dissertation 
will be focusing on this aspect. This will provide better and more novel research 
contributions compared to exploring functional properties. 
Furthermore, our survey of the visual analysis techniques used shows that the 
majority of existing works use some form of basic image diffing or features. This shows that it would be novel and potentially useful to investigate more fine-grained level of visual analysis that examines fine-grained visual details. Accordingly, this unexplored approach will be the basis of the techniques proposed in the remainder of the dissertation.}  


\balance
