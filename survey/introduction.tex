% !TEX root =  manuscript.tex
%\IEEEraisesectionheading{
	\section{Introduction}\label{sec:introduction}
%}


% \IEEEPARstart{S}{oftware} engineering (SE) is the application of
% a systematic, disciplined, quantifiable approach to
% the development, operation, and maintenance of 
% software~\cite{IEEEComputerSociety:2014:GSE:2616205}. 
%\IEEEPARstart{A}{ll} 
All areas of the software engineering (SE) lifecycle, 
such as requirements, design, development, and testing, 
often have the ultimate goal of contributing to a
fundamental product of software engineering: the source code.
Accordingly, a wide range of software engineering activities have
typically revolved around the source code,
whether to improve its quality, reliability, maintainability,
 or increase developers' productivity.
A relatively more recent, and scarcely explored, 
alternative is the adoption of a 
\emph{visual analysis} perspective.
This approach aims to extract, analyze, or process visual aspects
pertaining to the software, often using computer vision techniques. 
The objective is still focused on solving a software engineering problem, 
but is achieved via analyzing visual aspects of the software instead of relying 
exclusively on the source code.
As an example, a typical visual analysis  
might involve comparing a couple of screenshot images in order  
to compare or analyze two graphical user interfaces (GUI)
for testing purposes. 

Visual analysis techniques have yielded promising results 
in developing robust and accurate solutions
for various tasks.
For instance, they have been successfully adopted
to improve regression testing of GUIs
~\cite{Chang-2010-CHI, Alegroth-2013-ICST, Lin-2014-TSE},
to identify cross-browser incompatibilities in web pages
~\cite{Semenenko-2013-ICSM,Choudhary-2013-ICSE,Selay-2014-DICTA}, to 
perform bug detection and automated program repair~\cite{Mahajan-2014-ASE, Stocco-2018-FSE}, 
or to simplify software requirements modelling
~\cite{Li-2010-CHI, Scharf-2013-ICSE}.

In this chapter, we survey the literature on 
the use of visual analysis in performing 
software engineering tasks. 
Our work highlights visual analysis techniques and perspectives
of addressing research topics in software engineering,
what benefits they may provide compared to existing approaches,
and what limitations they might bear.
We believe this can be helpful in
providing a distilled and concise overview of 
visual approaches in software engineering, 
building a concrete understanding of the 
advances made, and synthesizing insights  
regarding future directions for the research community. 
We conducted the survey by formulating a number of
research questions to fulfill the goal of the study;
we then proceeded by systematically collecting
a pool of publications, and applied a number of
inclusion and exclusion criteria.
Subsequently, we analyzed and synthesized the collected papers
by taking into account a number of dimensions,
such as what area of software engineering (e.g., testing, maintenance)
they brought benefit to, 
what specific task is being addressed (e.g., regression testing),
what specific computer vision (CV) techniques have been used,
and what is the rationale for their adoption.


%\newpage