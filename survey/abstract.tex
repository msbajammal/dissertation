\IEEEtitleabstractindextext{%
\begin{abstract}
Software engineering (SE) research has traditionally revolved 
around engineering the source code. % of the software.
However, novel approaches that analyze 
software through computer vision have been increasingly adopted in SE.
%These techniques are being increasingly adopted 
%in SE research as they provide a complementary 
%perspective of the software that cannot be 
%obtained from the source code alone.
These approaches allow analyzing the software
from a different complementary perspective 
other than the source code,
and they are used to either complement 
existing source code-based methods,
or to overcome their limitations.
%
The goal of this manuscript is to survey
the use of computer vision techniques in SE
with the aim of assessing their potential 
in advancing the field of SE research. 
We examined an extensive body of literature 
from top-tier SE venues, as well as venues 
from closely related fields (machine learning, 
computer vision, and human-computer interaction). 
Our inclusion criteria targeted papers applying 
computer vision techniques that address problems 
related to any area of SE. 
We collected an initial pool of \initialPoolSize papers,
from which we obtained 66 
%\ali{the rebuttal says 64, which is it 66 or 64? double check!} 
%\andrea{it's 66}
final relevant papers
covering a variety of SE areas. 
We analyzed what computer vision techniques 
have been adopted or designed,
for what reasons, how they are used, 
what benefits they provide,
and how they are evaluated. 
Our findings highlight that visual approaches have been adopted in a wide variety of SE tasks,  
predominantly for effectively tackling
software analysis and testing challenges in the web and mobile domains.
The results also show a rapid growth trend 
of the use of computer vision techniques in SE research.
%We also summarize the main contributions,
%as well as their limitations, together with some directions for future work.
\end{abstract}

% Note that keywords are not normally used for peerreview papers.
\begin{IEEEkeywords}
Computer Vision, Software Engineering, Survey.
\end{IEEEkeywords}}

