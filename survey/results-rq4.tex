% !TEX root =  manuscript.tex
\subsection{Evaluation and Challenges (RQ4)}


\header{Evaluation Methods}

We systematically studied the selected papers to 
understand the most common methods used for 
evaluating the proposed visual approaches.
\Cref{table:evaluation-techniques} (Evaluation 
Methods) depicts the results: the evaluation 
methods presented are not necessarily disjoint,
since one technique can for instance be evaluated 
with respect to its performance (i.e., running time), 
and its accuracy calculated by the judgment of human participants.

In more than half of the works (53\%), the 
evaluation methods were performed using 
wide-spread effectiveness assessment 
measures such as precision and recall
~\cite{Choudhary-2010-ICSM, Choudhary-2012-ICST, 
Choudhary-2013-ICSE, Lin-2014-TSE, Nguyen-2015-ASE, 
Mahajan-2015-ICST, Hori-2015-SEKE, Feng-2016-ASE, 
Patric-2016-ASE, Wan-2017-STVR, He-2016-ICWS, 
Bao-2017-EMSE, Chen-2017-IUI, Reiss-2018-ASEj, 
Kirac-2018-JSS, canvas_icst2018, Kuchta-2018-EMSE, 
Xu-2018-TOIT}.

Another significant body of work measured 
the manual effort saved by the visual approach 
with respect to the humanly performed 
task~\cite{Delamaro-2011-STVR, Li-2010-CHI, 
Li-2010-CHI, Semenenko-2013-ICSM, Mahajan-2015-ICST, 
Feng-2016-ASE, Chen-2017-IUI, Leotta-2018-STVR, 
Reiss-2018-ASEj}. 
Moreover, several works also evaluated the 
performance of the proposed approaches, 
usually measuring the running time on common 
developer platforms (i.e., mid-level 
notebooks)~\cite{Nguyen-2015-ASE, Selay-2014-DICTA, 
Hori-2015-SEKE, Patric-2016-ASE, Wan-2017-STVR, 
Bao-2017-EMSE, Reiss-2018-ASEj, Kirac-2018-JSS}.

Performance is potentially considered as 
important as the accuracy because image analysis 
techniques are deemed as being computationally 
expensive. Thus, their application and use must 
carefully evaluate the overhead imposed by 
these visual approaches over the most general 
SE tool being developed. Initially, this might 
not favor their adoption if compared to other 
less computationally-intensive approaches.
However, the authors report that the time 
taken by their proposed techniques are all 
reasonable in common processing environments, 
suggesting that execution time should not be 
considered as a deterrent decision criterion when 
adopting a visual approach to support a SE task.

In several works, the human expertise was used to 
assess the output of the visual 
techniques~\cite{Mahajan-2015-ICST, Hori-2015-SEKE, 
Mahajan-2016-ICST, Kuchta-2018-EMSE, Xu-2018-TOIT, 
Ponzanelli-2016-ICSE, Choudhary-2010-ICSM, 
Choudhary-2012-ICST, Choudhary-2013-ICSE}.
As an example, \citet{Mahajan-2015-ICST} examine 
the efficacy of WebSee (a tool that identifies 
presentational failures in HTML pages)
through manual investigation of its output
when it is executed on 253 automatically-generated 
test cases.
This is done to see whether WebSee is able to 
correctly identify the faulty HTML elements
seeded when generating the test cases.
While we were expecting more human judgment 
involved in the evaluation of the selected papers, 
its application depends largely on the context 
and the problem being solved. 

Four works do not propose a quantitative empirical 
evaluation of the proposed technique, but 
rather present them by means of use cases.
A closer look at these publications revealed 
that two of them are short papers~\cite{Alegroth-2013-ICST, Scharf-2013-ICSE},
where having a more thorough evaluation is not 
mandatory. 
The other two papers~\cite{Burg-2015-UIST, 
Deka-2016-UIST} were published at the ACM Symposium 
on User Interface Software and Technology (UIST), 
in which use case demonstrations appear to be more frequent.
Four works included an industrial 
evaluation~\cite{Li-2010-CHI, Mahajan-2015-ICST, 
Amalfitano-2014-WISE, Kuchta-2018-EMSE} where the work is evaluated 
through feedback or measurements in an industrial setting.

% !TEX root =  manuscript.tex
\begin{table}
\caption{Evaluation methods and challenges.}
\centering
\setlength{\tabcolsep}{8pt}
%\renewcommand{\arraystretch}{3.2}
%\revised{0.5\textwidth}{
\begin{tabular}{lp{6cm}}
\toprule
% & \textbf{Studies} \\		
% \midrule											
\textbf{\textit{Evaluation Methods}} & \\ [0.5ex]

\quad Accuracy Measures & \cite{Fails-2003-CHI, Chen-2018-ICSE, Zhao-2019-ICSE,Sun-2018-ICML,Swearngin-2019-CHI,Wu-2020-CHI, Stocco-2018-FSE, Tanno-2018-ICSTW, bajammal2018generating, Moran-2018-ASE, Moran-ICSE-2018, Moran-TSE-2018, Natarajan-2018-MOBILESoft, Osman-2018-SEAA, Xiao-2019-ICSE, Huang-2019-CHI, Choudhary-2010-ICSM, Choudhary-2012-ICST, Choudhary-2013-ICSE, Lin-2014-TSE, Nguyen-2015-ASE, Mahajan-2015-ICST, Hori-2015-SEKE, Feng-2016-ASE, Patric-2016-ASE, Wan-2017-STVR, He-2016-ICWS, Bao-2017-EMSE, Chen-2017-IUI, Reiss-2018-ASEj, Kirac-2018-JSS, canvas_icst2018, Kuchta-2018-EMSE, Xu-2018-TOIT} \\

\quad Comparison against manual work & \cite{Zheng-2009-CHI,Liang-2013-UIST,Yuan-2020-CHI,  Tanno-2018-ICSTW, bajammal2018generating, Moran-2018-ASE, Moran-ICSE-2018, Moran-TSE-2018, Delamaro-2011-STVR, Li-2010-CHI, Li-2010-CHI, Semenenko-2013-ICSM, Mahajan-2015-ICST, Feng-2016-ASE, Chen-2017-IUI, Leotta-2018-STVR, Reiss-2018-ASEj} \\

\quad Survey or manual validation of results & \cite{Liang-2013-UIST,Wu-2017-CSCW,Chen-2018-ICSE,Lim-2018-UIST,Yu-2019-ASE,Wu-2020-CHI,  Bao-2018-TSE, Huang-2019-CHI, Caetano-2002-AAAI, Mahajan-2015-ICST, Hori-2015-SEKE, Mahajan-2016-ICST, Kuchta-2018-EMSE, Xu-2018-TOIT, Ponzanelli-2016-ICSE, Choudhary-2010-ICSM, Choudhary-2012-ICST, Choudhary-2013-ICSE} \\

\quad Time comparison & \cite{Fails-2003-CHI,  Chen-2018-ICSE,  Bao-2018-TSE, Stocco-2018-FSE, Tanno-2018-ICSTW, Natarajan-2018-MOBILESoft, Caetano-2002-AAAI, Nguyen-2015-ASE, Selay-2014-DICTA, Hori-2015-SEKE, Patric-2016-ASE, Wan-2017-STVR, Bao-2017-EMSE, Reiss-2018-ASEj, Kirac-2018-JSS} \\

\quad Comparison against competitor approaches & \cite{Stocco-2018-FSE, Coyette-2007-INTERACT, Caetano-2002-AAAI, Alegroth-2013-ICST, Lin-2014-TSE, Mahajan-2016-ICST, Deka-2017-UIST, He-2016-ICWS, Zhang-2017-ASE, Xu-2018-TOIT} \\

\quad Comparison to original design/output & \cite{Nguyen-2015-ASE, Patric-2016-ASE, Leotta-2018-STVR, canvas_icst2018, Wan-2017-STVR} \\

\quad Use case demonstration & \cite{Givens-2013-ICSE, Bao-2015-ICSE, Scharf-2013-ICSE, Alegroth-2013-ICST, Burg-2015-UIST, Deka-2016-UIST} \\
\quad Industry context/feedback & \cite{Bao-2018-TSE, Huang-2019-CHI, Coyette-2007-INTERACT, Li-2010-CHI, Mahajan-2015-ICST, Amalfitano-2014-WISE, Kuchta-2018-EMSE} \\
\quad Comparison with random/brute force & \cite{Mahajan-2014-ASE, Selay-2014-DICTA, Feng-2016-ASE, Patric-2016-ASE} \\
\quad Seeded faults & \cite{Mahajan-2014-ASE, Mahajan-2015-ICST, Mahajan-2016-ICST, Patric-2016-ASE} \\
\quad Differential comparisons & \cite{Chang-2010-CHI, Mahajan-2016-ICST} \\
\quad Line/state coverage & \cite{Zhang-2017-ASE, Deka-2017-UIST} \\
\quad Comparison on different physical devices & \cite{Lin-2014-TSE,Reinecke-2016-CHI} \\ [1ex]
		
\textbf{\textit{Challenges}} & \\	
		
%\quad {animations} & \cite{Bajammal-2018-ICST,Chang-2010-CHI,Choudhary-2010-ICSM,} \\

\quad {Noise} & \cite{Fails-2003-CHI,Bao-2015-ICSE,Reinecke-2016-CHI,   Stocco-2018-FSE, Tanno-2018-ICSTW, Moran-2018-ASE, Moran-ICSE-2018, Moran-TSE-2018, Natarajan-2018-MOBILESoft, Huang-2019-CHI, Caetano-2002-AAAI, Coyette-2007-INTERACT, Ponzanelli-2016-ICSE,Reiss-2018-ASEj,Bao-2017-EMSE,Nguyen-2015-ASE,Kuchta-2018-EMSE,Leotta-2018-STVR,Li-2010-CHI,Lin-2014-TSE,Kirac-2018-JSS} \\

\quad {Dynamicity} & \cite{Zheng-2009-CHI,Liang-2013-UIST,Lim-2018-UIST,  Bao-2018-TSE, Choudhary-2010-ICSM, Patric-2016-ASE,Wan-2017-STVR,canvas_icst2018,Chang-2010-CHI} \\

%\quad {sensitivity} & \cite{Chang-2010-CHI,Kirac-2018-JSS,Leotta-2018-STVR,Kuchta-2018-EMSE,Li-2010-CHI,Lin-2014-TSE} \\

\quad {Recognition} & \cite{Fails-2003-CHI,Chen-2018-ICSE,Zhao-2019-ICSE,Sun-2018-ICML,Swearngin-2019-CHI,Wu-2020-CHI,Yuan-2020-CHI,Givens-2013-ICSE,Bao-2015-ICSE,Wu-2017-CSCW,Yu-2019-ASE,   Zhang-2017-ASE, Bao-2018-TSE, Stocco-2018-FSE, Tanno-2018-ICSTW, Moran-2018-ASE, Moran-ICSE-2018, Moran-TSE-2018, Natarajan-2018-MOBILESoft, Osman-2018-SEAA, Xiao-2019-ICSE, Huang-2019-CHI, Caetano-2002-AAAI, Coyette-2007-INTERACT, Scharf-2013-ICSE,Bao-2017-EMSE,Kirac-2018-JSS,Kuchta-2018-EMSE} \\

%\quad {fading} & \cite{Chang-2010-CHI,Leotta-2018-STVR} \\
%
%\quad {scrolling} & \cite{Choudhary-2010-ICSM,Bao-2017-EMSE} \\

\quad{Visibility} & \cite{Lim-2018-UIST, Zhang-2017-ASE, Bao-2018-TSE, Chang-2010-CHI,Leotta-2018-STVR,Choudhary-2010-ICSM,Bao-2017-EMSE} \\

%\quad {dynamicity} & \cite{Choudhary-2010-ICSM, Patric-2016-ASE, Wan-2017-STVR} \\
		
\bottomrule
\end{tabular}
%}
\label{table:evaluation-techniques}
\end{table}



Finally, 
we noticed diversity in the evaluation methods of techniques aimed at solving similar problems. 
For example, several works have been proposed to identify some sort of \textit{visual defects} (e.g., all the approaches dealing with XBI). 
Four of these techniques~\cite{Mahajan-2014-ASE, Mahajan-2015-ICST, Mahajan-2016-ICST, Patric-2016-ASE} use ``seeded faults'' as an approach for constructing test oracles, whereas others take a different evaluation path. 
For instance, \citet{Choudhary-2013-ICSE} manually count all XBIs detected by different tools (i.e., their agreement)
as an upper bound for the number of issues that an XBI detection tool should identify, and use this number to compute the recall of X-PERT. As a further evaluation, authors could have been also evaluating their technique using seeded XBIs
to broaden the evaluation. 
Conversely, the works that only use seeded faults could use the agreement of multiple fault detectors to compute recall. 
This scenario essentially suggests that the evaluation of some of the proposed techniques in the literature can be substantially improved by leveraging the evaluation approaches from other techniques that aim at solving similar problems. 

\header{Challenges and Limitations}

Evaluations also exposed the main challenges and limitations that authors faced while developing their visual-based solutions. 
Unfortunately, a subset of the papers (41\%) either do not explicitly discuss drawbacks for the visual approach being evaluated, or do not provide concrete failing examples. On the other hand, the majority of the papers (59\%) do report the challenges encountered during the experimentations, summarized in \Cref{table:evaluation-techniques} (Challenges),  which we describe next. 

\header{Noise} 
The most recurring technical issue that inhibited the correct functioning of visual approaches concerns the \textit{noise} present in the visual artifacts~\cite{Ponzanelli-2016-ICSE,Reiss-2018-ASEj,Bao-2017-EMSE,Nguyen-2015-ASE,Kuchta-2018-EMSE,Leotta-2018-STVR,Li-2010-CHI,Lin-2014-TSE}. 
Noise refers to the presence of other random or overlapping items in the background of the visual artifact that prevents, for instance, the correct detection of the target object. Hence, before applying the desired visual method, a pre-processing technique is often used to clear the noise out of the artifact. 

As an example,~\citet{Ponzanelli-2016-ICSE} apply OCR to detect source code in frames sampled from video tutorials. However, the effectiveness of OCR fluctuates dramatically if the considered frame's background contains non-pertinent information. As a solution, the authors utilize two additional visual techniques---shape detection and frame segmentation---in order to focus OCR towards the area of interest.

Another more specific noise-related limitation pertains to the \textit{sensitivity} of the visual approach to theme/surroundings changes, lighting conditions, and reflections~\cite{Chang-2010-CHI,Kirac-2018-JSS,Leotta-2018-STVR,Kuchta-2018-EMSE,Li-2010-CHI,Lin-2014-TSE}. A possible mitigation concerns using threshold parameters to limit their detrimental effect. 

\header{Dynamicity} Highly-variable visual artifacts are also a major challenge for the design and development of reliable visual approaches. 

One source of fragility is due to animations~\cite{canvas_icst2018,Chang-2010-CHI,Choudhary-2010-ICSM}. Indeed, a continuously-changing visual artifact (e.g., a video frame) represents a major limitation for many one-shot techniques. As a possible solution, such techniques would need to be applied repeatedly, similarly as real-time domains, or, in extreme cases, re-designed from scratch to meet the new domain requirements. 

Another source of fragility is due to web advertisements~\cite{Wan-2017-STVR}.
An ad is rendered as a dynamic component within a more static container (i.e., a web page),
the visual representation of which usually changes across consecutive loads of the web page or over different time stamps.
Visual methods need to isolate the dynamics of web pages to avoid erroneous behaviour or false positives.  

\header{Recognition}
Another challenge of visual approaches that emerged from our study concerns (1)~accurately identifying small imperceptible differences between images and (2)~recognizing complex user-defined actions such as manually inserted strokes or handwriting. 

For example, a source of false positives in \textsc{VISOR}~\cite{Kirac-2018-JSS}, an image comparison technique used for black-box testing of digital TVs, is the tiny differences that occur when rendering accent characters on the screen (e.g., cedillas). 
Typically, the choice on whether such differences should be considered as defects or not is left to human's perception. 
%
Two other works experienced recognition issues, but not at the same pixel-level as \textsc{VISOR}. \citet{Bao-2017-EMSE} discuss problems in detecting visual fine-grained developers' actions from videos, such as code editing, text selection, and window scrolling. Similarly,~\citet{Scharf-2013-ICSE} mention issues in detecting handwriting in manually-produced sketches. 

These findings demonstrate how complex is the set of visual differences that can emerge while comparing two images, and how vast and multifaceted is the set of input actions that are possible on a user interface. Indeed, when a visual method is used to actively support the \emph{end user}, they arguably need to recognize a broader set of inputs than those they would receive from another algorithm, if they were executed in a software controlled environment. This poses new challenges to the development of robust visual approaches for aiding SE.


\header{Visibility}
At last, in four works, having the visual artifact present in the main rendering area is mentioned as a requirement for the visual technique to effectively fulfill its task.
This requirement can be violated by, e.g., \textit{fading}~\cite{Chang-2010-CHI,Leotta-2018-STVR} and \textit{scrolling}~\cite{Choudhary-2010-ICSM,Bao-2017-EMSE}.

As an example, \citet{Leotta-2018-STVR}'s tool \textsc{PESTO} uses template matching to detect the optimal visual locator corresponding to a web element on the page. However, the web element of interest might be outside the actual rendered web page. This happens when a long and complex form with multiple fields cannot be entirely visualized within the main screen. The authors propose an engineering workaround to solve issues related to scrolling: if the visual artifact being searched is not immediately displayed, the tool scrolls the page down automatically and the template matching is repeated.


