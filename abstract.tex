%% The following is a directive for TeXShop to indicate the main file
%%!TEX root = diss.tex

\chapter{Abstract}

Non-functional software properties capture qualitative and generic aspects about software. 
Such aspects are often high level and more semantic compared to the more precise and quantitative functional properties or requirements, and therefore have been more difficult 
to analyze and automate.  
A scarcely explored, and potentially useful, alternative 
paradigm is the adoption of what might be referred to as a visual analysis 
approach to software engineering, which involves extracting or analyzing 
visual information pertaining to the software, with the 
objective of addressing software engineering problems.  

The goal of the work presented in this dissertation is to improve 
non-functional web UI properties using automated visual analysis. 
We focus on particular problems of testability, accessibility, and maintainability 
because they have not been amenable to automation so far.  
First, we improve testability by converting the inherently non-testable 
web canvas elements into testable ones. 
The automated technique is based on visually analyzing the structure and  
properties of the canvas contents, then augmenting them 
into the canvas element to make it testable. 
Then, we propose an approach to test semantic accessibility.
It is based on visually analyzing various regions of the page and then inferring 
any associated semantic roles, after which the UI markup is examined to 
assert the presence of the roles. 
Next, we introduce an automated technique for addressing the common problem 
of inaccessible web form labeling. 
It is based on constructing visual cues from the form, 
then solving for the optimal labeling associations, which are finally augmented 
into the inaccessible web forms to make them accessible. 
Finally, we present a UI component generation technique to improve maintainability. 
The technique first detects visual patterns in the UI, then combines subsets of these 
patterns into a shared template, which is finally formulated as a UI component.  
Our evaluations show that the proposed techniques are able to carry the inferences, 
analyses, and tests in an accurate and effective manner.  



%\ifgpscopy
%  This document was typeset in \texttt{gpscopy} mode.
%\else
%  This document was typeset in non-\texttt{gpscopy} mode.
%\fi

% Consider placing version information if you circulate multiple drafts
%\vfill
%\begin{center}
%\begin{sf}
%\fbox{Revision: \today}
%\end{sf}
%\end{center}
