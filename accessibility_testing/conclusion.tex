\section{Conclusion}

Software accessibility is the notion of building software that 
is usable by users with disabilities. Traditionally, software 
accessibility has often been an afterthought or a nice to have 
optional feature. However, software accessibility is increasingly 
becoming a legal requirement that must be satisfied. 
While some tools exist to perform basic forms of accessibility checks, 
they focus on syntactic checks, as opposed to checking the more critical 
high level semantic accessibility features that users with disabilities 
rely on. 
In this chapter, we proposed an approach that automates web accessibility 
testing from a semantic perspective. It analyzes web pages using a combination 
of visual analysis, supervised machine learning, and natural language 
processing, and infers the semantic groupings present in the page 
and their semantic roles. 
It then asserts whether the page's markup matches the inferred semantics. 
We evaluated our approach on 30 real-world websites and assessed the accuracy 
of semantic inference as well as its ability to detect accessibility failures. 
The results show, on average, an F-measure of 87\% for 
inferring semantic groupings, and an accessibility failures detection accuracy 
of 85\%.

