\section{Related Work}

\header{Accessibility attribute checkers}
Existing approaches related to accessibility testing focus on checking 
syntactical attributes. 
Eler et al.~\cite{eler2018automated} and Patil et al.~\cite{patil2016enhanced} 
check for missing or wrong UI attributes in Android apps, 
such as missing alternative 
text attributes in images, or color attribute values 
below a certain threshold. 
Similar checks are also used in other tools such as 
Google's Accessibility Scanner~\cite{goog_scanner}, 
WAVE~\cite{wave_tool}, and ASLint~\cite{aslint_tool}.
The aforementioned works focus on syntactic checks, 
as opposed to the proposed approach in this work which 
checks for high-level aspects such as page structure and semantic 
landmarks, which are the most important ARIA roles that users 
with disabilities rely on~\cite{2019users_survey}. 

\header{Accessibility guidelines}
The majority of existing work lies within the 
accessibility research community 
rather than software engineering.
This research area involves studying certain categories of websites 
(e.g., airline websites~\cite{agrawal2019evaluating,dominguez2018website}, 
education portals~\cite{kimmons2017open}, 
other categories~\cite{bhagat2019evaluation,ross2018examining,snider2020accessibility,serra2015accessibility}) or 
certain platforms (e.g., Android~\cite{alshayban2020accessibility,park2014toward,yan2019current}), 
and then focusing on \emph{manually observing} how non-sighted users 
would use those apps or websites in order to 
identify any patterns or trends in accessibility, with the 
purpose of publishing improved accessibility guidelines. 
Another line of work focuses on researching software development 
\emph{best practices} and how do they impact the accessibility 
of the end product. 
For instance, Sanchez et al.~\cite{sanchez2017method} and 
Bai et al.~\cite{bai2018categorization, bai2019methods} 
examine development practices in 
agile teams working on accessible software, 
with the goal of proposing a guideline 
for better agile practices. 
Krainz et al.~\cite{krainz2018can} investigates 
the impact of a model-driven approach to development 
on the accessibility of the created product.
None of the aforementioned works, however, is concerned 
with developing an automated approach to test accessibility.
Instead, their focus is researching best practices or guidelines for
developers and designers.

\header{Visual analysis}
There exist a few techniques that analyze web applications from a visual perspective.
Choudhary et al.~\cite{choudhary2012crosscheck} propose an approach that detects 
cross-browser compatibility by examining visual differences between the same app 
running in multiple browsers.
Burg et al.~\cite{burg2015explaining} present a tool that helps developers 
understand the behavior of front-end apps. It allows developers to specify 
the element they are interested in, then tracks that element for any 
visual changes to understand code behavior. 
Bajammal et al.~\cite{bajammal2018generating} propose an approach to generate reusable web 
components by analyzing design mockups.
In contrast to our work, none of these works are related to accessibility. 



