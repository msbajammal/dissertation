%% The following is a directive for TeXShop to indicate the main file
%%!TEX root = diss.tex

%% https://www.grad.ubc.ca/current-students/dissertation-thesis-preparation/preliminary-pages
%% 
%% LAY SUMMARY Effective May 2017, all theses and dissertations must
%% include a lay summary.  The lay or public summary explains the key
%% goals and contributions of the research/scholarly work in terms that
%% can be understood by the general public. It must not exceed 150
%% words in length.

\chapter{Lay Summary}

%The analysis and testing of software has traditionally revolved around 
%software's code. While this is useful in many cases, some aspects 
%of web apps have been difficult to analyze this way. 
%Examples of such aspects include whether or not the app can be tested, 
%how easy it is for users with disabilities to use the app, 
%and how easy it is to maintain the app. 
%The work in this thesis improves these aspects of web apps   
%through the use of automated visual analysis. This refers to a scarcely 
%explored, but potentially beneficial, approach that looks at the app  
%visually to get a complementary perspective. 
%The proposed techniques capture a variety of visual aspects of the app and 
%then uses them to achieve various analysis tasks. 
%Our evaluations show that these techniques improve the extent to which an app  
%can be tested, made more accessible, and easier to maintain. 
  
When engineers analyze or test qualitative aspects of software, such as how 
accessible is the software, they often have to do so manually because there 
are not many techniques that can automatically help them in their testing.  
The work in this dissertation provides a solution to such cases. 
We propose techniques to analyze visual information about web apps in order 
to obtain useful and high level information about the app. 
Using this analysis approach, the dissertation demonstrates how can we make web apps  
easier to test, made more accessible for users with disabilities, and made easier 
to maintain.  

 