% !TEX root =  paper.tex

\usepackage{booktabs}
\usepackage[caption=false,font=normalsize,labelfont=sf,textfont=sf]{subfig}
%\usepackage{cite}
\usepackage{hyperref}
\usepackage{array}
\usepackage{threeparttable}
\usepackage{enumitem}    
\usepackage[greek,english]{babel}
\usepackage{ifthen}
\usepackage{xspace}
\usepackage{fancybox}
\usepackage{marginnote}
\usepackage{tcolorbox}
\usepackage{multirow}
\usepackage{mathtools}
\usepackage{algpseudocode, algorithm, algorithmicx}
\usepackage{color,soul}
\usepackage{graphicx}
\usepackage{amsmath, amssymb}
\usepackage{tikz}
\usepackage{cleveref}
\usepackage{hhline}
\usepackage{balance}


\newboolean{showcomments} 
\setboolean{showcomments}{true}
\ifthenelse{\boolean{showcomments}}
{\newcommand{\nb}[2]{
		\fbox{\bfseries\sffamily\scriptsize#1}%
		{\sf\footnotesize$\blacktriangleright$\textcolor{blue}{\hl{#2}}$\blacktriangleleft$}
	}
}
{\newcommand{\nb}[2]{}}

\definecolor{circled-color}{gray}{0.15}
\newcommand*\circled[1]{\tikz[inner sep=.1ex,baseline=-.75ex] \node[circle,draw,color=white,fill=circled-color] {#1};}


\newcommand{\VizElem}{visual element}
\newcommand{\VE}{VE}
\DeclareMathOperator*{\argmax}{argmax}
\DeclareMathOperator*{\argmin}{argmin}

%\usepackage{mathptmx}

%\let\temp\rmdefault
%\usepackage{mathpazo}
%\let\rmdefault\temp

\newcommand\ali[1]{\nb{Ali}{#1}}
\newcommand\mohammad[1]{\nb{Mohammad}{#1}}
\newcommand\davood[1]{\nb{Davood}{#1}}

\newcommand{\todo}[1]{\textcolor{magenta}{\nb{TODO:}{#1}}}

%\newcommand{\MyParagraph}[1]{\par\smallskip\noindent\textbf{#1}}

\newcommand{\header}[1]{\par\vspace{-1mm}\medskip\noindent\textbf{#1.}}

\newcommand{\hide}[1]{}

\newcommand{\code}[1]{\texttt{\fontsize{9.5}{11}\selectfont #1}}
\newcommand{\smcode}[1]{\texttt{\fontsize{7.5}{8}\selectfont #1}}

\definecolor{findingsbox-bg-color}{gray}{0.90}
\newtcbox{\findingsbox}{colback=findingsbox-bg-color, boxrule=0.2pt, arc=2pt, boxsep=0pt, left=5pt, right=5pt, top=5pt, bottom=5pt}
\newcommand{\findings}[2] {
	\vspace{5pt}
	\noindent
	\findingsbox{
		\begin{minipage}{.95\linewidth}
			\textbf{#1}: #2
		\end{minipage}
	}
}

\newtheorem{defn}{Definition}

% Define tool names, etc. here
\newcommand{\html}{\textsc{HTML}\xspace}
\newcommand{\css}{\textsc{CSS}\xspace}
\newcommand{\javascript}{\textsc{JavaScript}\xspace}
\newcommand{\react}{\textsc{React}\xspace}
\newcommand{\angular}{\textsc{Angular}\xspace}
\newcommand{\nicad}{\textsc{NiCad}\xspace}
\newcommand{\xpath}{\textsc{XPath}\xspace}
\newcommand{\dom}{\textsc{DOM}\xspace}
\newcommand{\cssdev}{\textsc{CSSDev}\xspace}

\newcommand{\model}{Component Intermediate Model\xspace}
\newcommand{\mappedset}{Mapping Nodes Set\xspace}

\newcommand{\toolname}{\textsc{VizMod}\xspace}

%\newcommand{\numberOfEmails}{\hl{X}\xspace}
\newcommand{\numberOfTemplates}{five\xspace}
\newcommand{\numberOfParticipants}{five\xspace}
\newcommand{\numberOfComponents}{25\xspace}
\newcommand{\totalNumberOfComponentInstances}{120\xspace}
\newcommand{\precision}{94\%\xspace}
\newcommand{\recall}{75\%\xspace}
\newcommand{\sizeReductionMin}{6.01\%\xspace}
\newcommand{\sizeReductionmax}{19.34\%\xspace}




% These are used for the rebuttal 
\newboolean{showreviewhints} 
\setboolean{showreviewhints}{true}
\ifthenelse{\boolean{showreviewhints}}{
\definecolor{reviewcolor}{RGB}{0,0,255}
	\newcommand{\revised}[2]{\setlength{\marginparwidth}{1.05cm}\marginnote{\color{reviewcolor}{\fbox{\parbox{\dimexpr\linewidth-1\fboxsep-2\fboxrule}{#2}}}}{\color{reviewcolor}{#1}}}
}
{\newcommand{\revised}[2]{#1}}
