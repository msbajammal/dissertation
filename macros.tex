%% The following is a directive for TeXShop to indicate the main file
%%!TEX root = diss.tex




% TSE paper macros
\usepackage{cite}
\usepackage{booktabs} % For formal tables
\usepackage{soul}
\usepackage[table,dvipsnames]{xcolor}
\usepackage{color}
\usepackage[utf8]{inputenc}
\usepackage{amssymb}
\usepackage{ifthen}
\usepackage{multirow}
\usepackage{caption}
\usepackage{listings}
\usepackage{hyperref}
\usepackage{url,moreverb,xspace}
\usepackage{array,graphicx}
\usepackage{balance}
\usepackage{pifont}
\usepackage{threeparttable}
\usepackage[numbers]{natbib}
\usepackage{etoolbox,siunitx}
\usepackage{enumitem}
\usepackage{tcolorbox}

\usepackage{longtable}

%\soulregister{\cite}
%\soulregister{\emph}
\sethlcolor{green}

%\usepackage{showframe}
\sisetup{output-decimal-marker={.},group-separator={,},group-minimum-digits=4,detect-family=true}

% MACROS
\newtheorem{defn}{Definition}

\newboolean{showcomments}
\setboolean{showcomments}{true}
\ifthenelse{\boolean{showcomments}}
{\newcommand{\nb}[2] {
		\fcolorbox{black}{gray!20}{\bfseries\sffamily\scriptsize#1:}
		{\sf\small$\blacktriangleright$\textit{#2}$\blacktriangleleft$}
	}
}
{\newcommand{\nb}[2]{}
}
\newcommand\mohammad[1]{\nb{Mohammad}{\hl{#1}}}
\newcommand\andrea[1]{\nb{Andrea}{\hl{#1}}}
\newcommand\davood[1]{\nb{Davood}{\hl{#1}}}
\newcommand\ali[1]{\nb{Ali}{\hl{#1}}}

% \tcbset{width=0.48\textwidth,boxrule=0pt,colback=Goldenrod,arc=0pt,auto outer arc,left=0pt,right=0pt,boxsep=0pt}
% \newcommand\revised[1]{
%   \begin{tcolorbox}
%     \color{OliveGreen} #1
%   \end{tcolorbox}
% }

% \definecolor{responseColor}{RGB}{0, 105, 60}
\definecolor{responseColor}{RGB}{255, 255, 255}

\newcommand{\hlcolor}{responseColor}
\newcommand{\revised}[2]{
	\colorbox{responseColor}{\parbox{#1}{\color{black} #2}}
}

% \newcommand{\changed}[1]{{\color{responseColor}{#1}}}
\newcommand{\changed}[1]{#1}

\newcommand{\header}[1]{\par\vspace{-1mm}\medskip\noindent\textbf{#1.}}

\newcommand{\technique}{CV\xspace}
\newcommand{\numberOfPapers}{66\xspace}
\newcommand{\initialPoolSize}{2,716\xspace}














% Canvas paper macros
\newcommand\textlcsc[1]{\textsc{\MakeLowercase{#1}}}

\usepackage{algorithm}
\usepackage{algorithmicx}
\usepackage{algpseudocode}
\usepackage{balance}

% correct bad hyphenation here
\hyphenation{op-tical net-works semi-conduc-tor}

\usepackage{graphicx}
\usepackage{listings}
\usepackage{color}
\usepackage{soul}  % for \hl{}
\usepackage{amsmath}
\usepackage{moreverb,xspace}
\usepackage{booktabs} % For formal tables
\usepackage{multirow}
\usepackage{cleveref}

\definecolor{lightgray}{rgb}{0.95, 0.95, 0.95}
\definecolor{darkgray}{rgb}{0.4, 0.4, 0.4}
%\definecolor{purple}{rgb}{0.65, 0.12, 0.82}
\definecolor{editorGray}{rgb}{0.95, 0.95, 0.95}
\definecolor{editorOcher}{rgb}{1, 0.2, 0} % #FF7F00 -> rgb(239, 169, 0)
\definecolor{editorGreen}{rgb}{0, 0.5, 0} % #007C00 -> rgb(0, 124, 0)
\definecolor{orange}{rgb}{1,0.45,0.13}      
\definecolor{olive}{rgb}{0.17,0.59,0.20}
\definecolor{brown}{rgb}{0.69,0.31,0.31}
\definecolor{purple}{rgb}{0.38,0.18,0.81}
\definecolor{lightblue}{rgb}{0.1,0.30,0.7}
\definecolor{lightred}{rgb}{1,0.4,0.5}


\lstdefinelanguage{javascript}{
	morekeywords={typeof, new, true, false, catch, function, return, null, catch, switch, var, if, in, while, do, else, case, break},
	morecomment=[s]{/*}{*/},
	morecomment=[l]//,
	morestring=[b]",
	morestring=[b]'
}

\lstset{
	%     backgroundcolor=\color{backcolour},   
	%     keywordstyle=\color{codegreen},
	%     numberstyle=\tiny\color{codegray},
	%     stringstyle=\color{codepurple},
	%     basicstyle={\sffamily},
	%     breakatwhitespace=false,         
	%     breaklines=true,                 
	%     captionpos=t,                    
	%     keepspaces=true,                 
	%     numbers=left,                    
	%     numbersep=5pt,                  
	%     showspaces=false,                
	%     showstringspaces=false,
	%     showtabs=false,                  
	%     tabsize=2
	% General design
	backgroundcolor=\color{editorGray},
	basicstyle={\scriptsize\ttfamily},   
	captionpos=b,
	frame=bt,
	% line-numbers
	xleftmargin={0.75cm},
	numbers=left,
	stepnumber=1,
	firstnumber=1,
	numberfirstline=true, 
	% Code design
	identifierstyle=\color{black}\ttfamily,
	keywordstyle=\color{lightblue}\bfseries\ttfamily,
	ndkeywordstyle=\color{editorGreen}\bfseries\ttfamily,
	stringstyle=\color{olive}\ttfamily,
	commentstyle=\color{brown}\ttfamily,
	% Code
	tabsize=2,
	showtabs=false,
	showspaces=false,
	showstringspaces=false,
	extendedchars=true,
	breaklines=true,
	% German umlauts
	literate=%
	{Ö}{{\"O}}1
	{Ä}{{\"A}}1
	{Ü}{{\"U}}1
	{ß}{{\ss}}1
	{ü}{{\"u}}1
	{ä}{{\"a}}1
	{ö}{{\"o}}1    
}

\newcommand{\tool}{\textsc{CanvaSure}\xspace}

\newcommand{\head}[1]{\par\smallskip\noindent\textbf{#1.}}











% Accessibility testing macros
% !TEX root =  paper.tex

% \usepackage{cite}
% \usepackage{amsmath,amssymb,amsfonts}
% \usepackage{algorithmic}
% \usepackage{graphicx}
% \usepackage{subcaption}
% \usepackage{textcomp}
% \usepackage{xcolor}
% \usepackage{multicol} % \columnbreak
% \usepackage{balance}
% \usepackage{enumitem}    
\usepackage[table]{xcolor}

\usepackage{tabularx} 
\usepackage{collcell}

% \usepackage[usenames,dvipsnames,svgnames,table,xcdraw]{xcolor}
% \usepackage[usenames,dvipsnames,svgnames]{xcolor}
% \usepackage{pgfplots}
% \pgfplotsset{compat=1.10}


\usepackage{booktabs}
\usepackage[caption=false,font=normalsize,labelfont=sf,textfont=sf]{subfig}
%\usepackage{cite}
% \usepackage{hyperref}
\usepackage{array}
\usepackage{threeparttable}
\usepackage{enumitem}    
%\usepackage[greek,english]{babel}
\usepackage{ifthen}
\usepackage{xspace}
\usepackage{fancybox}
\usepackage{marginnote}
\usepackage{tcolorbox}
\usepackage{multirow}
\usepackage{mathtools}
\usepackage{algpseudocode, algorithm, algorithmicx}
\usepackage{color}
\usepackage{soul}
\usepackage{graphicx}
\usepackage{amsmath, amssymb}
\usepackage{tikz}
%\usepackage{cleveref}
\usepackage{hhline}
\usepackage{balance}

\usepackage{listings}
\usepackage{parcolumns}
%\usepackage{cleveref}

\usepackage{array}

\usepackage{adjustbox}
\usepackage{flushend}
\usepackage[switch]{lineno}

\definecolor{light-gray}{gray}{0.85}
\newcommand{\code}[1]{\colorbox{light-gray}{\fontsize{9pt}{10pt}\texttt{#1}}}

\definecolor{circled-color}{gray}{0.15}
\newcommand*\circled[1]{\tikz[inner sep=.1ex,baseline=-.75ex] \node[circle,draw,color=white,fill=circled-color] {#1};}


\def\BibTeX{{\rm B\kern-.05em{\sc i\kern-.025em b}\kern-.08em
		T\kern-.1667em\lower.7ex\hbox{E}\kern-.125emX}}

%\newboolean{showcomments}
\setboolean{showcomments}{false}
\ifthenelse{\boolean{showcomments}}
{
	\definecolor{myyellow}{RGB}{255, 228, 26}
	\definecolor{myblue}{RGB}{50, 50, 220}
	\definecolor{myred}{RGB}{250, 10, 10}
%	\newcommand{\nb}[2]{
%		{\sf
%			\fcolorbox{myyellow}{orange}{\scriptsize\textbf{#1}}%
%			$\blacktriangleright$%
%			{\color{myred}\fontsize{7pt}{8pt}\selectfont\textbf{[[#2]]}}%
%		}%
%	}
}
{
%	\newcommand{\nb}[2]{}
}

%\newcommand{\mohammad}[1]{\nb{Mo}{#1}}
%\newcommand{\ali}[1]{\nb{Ali}{#1}}


\usepackage{color}


\definecolor{editorGray}{rgb}{0.95, 0.95, 0.95}
\definecolor{editorOcher}{rgb}{1, 0.5, 0} % #FF7F00 -> rgb(239, 169, 0)
\definecolor{editorGreen}{rgb}{0, 0.5, 0} % #007C00 -> rgb(0, 124, 0)

\lstdefinelanguage{JavaScript}{
	morekeywords={typeof, new, true, false, catch, function, return, null, catch, switch, var, if, in, while, do, else, case, break},
	morecomment=[s]{/*}{*/},
	morecomment=[l]//,
	morestring=[b]",
	morestring=[b]'
}
\lstdefinelanguage{HTML5}{
	language=html,
	sensitive=true, 
	alsoletter={<>=-},
	otherkeywords={
		% HTML tags
		<html>, <head>, <title>, </title>, <meta, />, </head>, <body>,
		<canvas, \/canvas>, <script>, </script>, </body>, </html>, <!, html>, <style>, </style>, ><
	},  
	ndkeywords={
		% General
		=,
		% HTML attributes
		charset=, id=, width=, height=,
		% CSS properties
		border:, transform:, -moz-transform:, transition-duration:, transition-property:, transition-timing-function:
	},  
	morecomment=[s]{<!--}{-->},
	tag=[s]
}

\lstdefinelanguage{CSS}{
	morekeywords={background,color,display,justify,content,font,weight,border,size,padding},
	morestring=[s]{:}{;},
	sensitive,
	morecomment=[s]{/*}{*/}
}

\lstset{%
	% Basic design
	backgroundcolor=\color{editorGray},
	basicstyle={\scriptsize\ttfamily},   
	frame=l,
	% Line numbers
	xleftmargin={0.75cm},
	numbers=left,
	stepnumber=1,
	firstnumber=1,
	numberfirstline=true,
	% Code design   
	keywordstyle=\color{blue}\bfseries,
	commentstyle=\color{darkgray}\ttfamily,
	ndkeywordstyle=\color{editorGreen}\bfseries,
	stringstyle=\color{editorOcher},
	% Code
	language=HTML5,
	alsolanguage=JavaScript,
	alsodigit={.:;},
	tabsize=2,
	showtabs=false,
	showspaces=false,
	showstringspaces=false,
	extendedchars=true,
	breaklines=true,        
	% Support for German umlauts
	literate=%
	{Ö}{{\"O}}1
	{Ä}{{\"A}}1
	{Ü}{{\"U}}1
	{ß}{{\ss}}1
	{ü}{{\"u}}1
	{ä}{{\"a}}1
	{ö}{{\"o}}1
}

\newcommand{\todo}[1]{\textcolor{magenta}{\nb{TODO:}{#1}}}
%\newcommand{\header}[1]{\par\smallskip\noindent\textbf{#1.}}

\newcommand{\html}{\textsc{HTML}\xspace}
\newcommand{\css}{\textsc{CSS}\xspace}
\newcommand{\javascript}{\textsc{JavaScript}\xspace}

\newcommand{\toolname}{\textsc{AxeRay}\xspace}

\newcommand{\UIelements}{UI elements}
\newcommand{\UIelement}{UI element}

\newcommand{\VizObjs}{Visual Objects}
\newcommand{\VizObj}{Visual Object}
\newcommand{\vizobj}{visual object}
\newcommand{\vizobjs}{visual objects}

















% Web form accessibility macros
% !TEX root =  paper.tex

		%\usepackage{cite}
		% \usepackage{amsmath,amssymb,amsfonts}
		% \usepackage{algorithmic}
		% \usepackage{graphicx}
		% \usepackage{subcaption}
		% \usepackage{textcomp}
		% \usepackage{xcolor}
		% \usepackage{multicol} % \columnbreak
		% \usepackage{balance}
		% \usepackage{enumitem}    
		%\usepackage[table]{xcolor}
		%\usepackage{nohyperref}
%\usepackage{xcolor}
%\usepackage{tabularx} 
%\usepackage{collcell}
		% \usepackage[usenames,dvipsnames,svgnames,table,xcdraw]{xcolor}
		% \usepackage[usenames,dvipsnames,svgnames]{xcolor}
		% \usepackage{pgfplots}
		% \pgfplotsset{compat=1.10}
%\usepackage{booktabs}
%\usepackage[caption=false,font=normalsize,labelfont=sf,textfont=sf]{subfig}
		%\usepackage{cite}
		% \usepackage{hyperref}
%\usepackage{array}
%\usepackage{threeparttable}
%\usepackage{enumitem}    
%\usepackage[greek,english]{babel}
%\usepackage{ifthen}
%\usepackage{xspace}
%\usepackage{fancybox}
%\usepackage{marginnote}
%\usepackage{tcolorbox}
%\usepackage{multirow}
%\usepackage{mathtools}
%\usepackage{algpseudocode, algorithm, algorithmicx}
%\usepackage{color}
%\usepackage{soul}
%\usepackage{graphicx}
%\usepackage{amsmath}
%\usepackage{tikz}
%\usepackage{cleveref}
%\usepackage{hhline}
%\usepackage{balance}
%\usepackage{listings}
%\usepackage{parcolumns}
%\usepackage{cleveref}
%\usepackage{array}
%\usepackage{adjustbox}
%\usepackage{flushend}
%\usepackage[switch]{lineno}
%\definecolor{light-gray}{gray}{0.95}
		%\newcommand{\code}[1]{\colorbox{light-gray}{\fontsize{9pt}{10pt}\texttt{#1}}}

%\definecolor{circled-color}{gray}{0.15}
		%\newcommand*\circled[1]{\tikz[inner sep=.1ex,baseline=-.75ex] \node[circle,draw,color=white,fill=circled-color] {#1};}


%\def\BibTeX{{\rm B\kern-.05em{\sc i\kern-.025em b}\kern-.08em
%		T\kern-.1667em\lower.7ex\hbox{E}\kern-.125emX}}
		%\newboolean{showcomments}
%\setboolean{showcomments}{true}
%\hypersetup{draft,bookmarks=false}

		%\ifthenelse{\boolean{showcomments}}
		%{
		%	\definecolor{myyellow}{RGB}{255, 228, 26}
		%	\definecolor{myblue}{RGB}{50, 50, 220}
		%	\newcommand{\nb}[2]{
		%		{\sf
		%			\fcolorbox{myyellow}{yellow}{\scriptsize\textbf{#1}}%
		%			$\blacktriangleright$%
		%			{\color{myblue}\fontsize{7pt}{8pt}\selectfont\textbf{#2}}%
		%		}%
		%	}
		%}
		%{
		%	\newcommand{\nb}[2]{}
		%}
		
		
		%\newcommand{\Mo}[1]{\nb{Mo}{#1}}
		%\newcommand{\Ali}[1]{\nb{Ali}{#1}}

%\usepackage{color}

%\definecolor{editorGray}{rgb}{0.95, 0.95, 0.95}
%\definecolor{editorOcher}{rgb}{1, 0.5, 0} % #FF7F00 -> rgb(239, 169, 0)
%\definecolor{editorGreen}{rgb}{0, 0.5, 0} % #007C00 -> rgb(0, 124, 0)

\lstdefinelanguage{JavaScript}{
	morekeywords={typeof, new, true, false, catch, function, return, null, catch, switch, var, if, in, while, do, else, case, break},
	morecomment=[s]{/*}{*/},
	morecomment=[l]//,
	morestring=[b]",
	morestring=[b]'
}
\lstdefinelanguage{HTML5}{
	language=html,
	sensitive=true, 
	alsoletter={<>=-},
	otherkeywords={
		% HTML tags
		<html>, <head>, <form>, </form>, <div>, </div>, <p>, </p>, 
		<span>, </span>, <input, <textarea, </textarea>, <select, </select>, 
		<option>, </option>, <title>, </title>, <meta, />, </head>, <body>,
		<canvas, \/canvas>, <script>, </script>, </body>, </html>, <!, html>, <section>, </section>
	},  
	ndkeywords={
		% General
		=,
		% HTML attributes
		charset=, id=, width=, height=, type=, value=, checked
		% CSS properties
		border:, transform:, -moz-transform:, transition-duration:, transition-property:, transition-timing-function:
	},  
	morecomment=[s]{<!--}{-->},
	tag=[s]
}

\lstdefinelanguage{CSS}{
	morekeywords={background,color,display,justify,content,font,weight,border,size,padding},
	morestring=[s]{:}{;},
	sensitive,
	morecomment=[s]{/*}{*/}
}
		%\newcommand{\todo}[1]{\textcolor{magenta}{\nb{TODO:}{#1}}}
		%\newcommand{\header}[1]{\par\smallskip\noindent\textbf{#1.}}
		%\newcommand{\toolname}{\textsc{AxeForm}\xspace}
		%\newcommand{\html}{\textsc{HTML}\xspace}
		%\newcommand{\css}{\textsc{CSS}\xspace}
		%\newcommand{\javascript}{\textsc{JavaScript}\xspace}
\lstset{%
	% Basic design
	backgroundcolor=\color{editorGray},
	basicstyle={\linespread{1.0}\scriptsize\ttfamily},   
	frame=l,
	% Line numbers
	xleftmargin={0.75cm},
	numbers=left,
	stepnumber=1,
	firstnumber=1,
	numberfirstline=true,
	% Code design   
	keywordstyle=\color{blue}\bfseries,
	commentstyle=\color{darkgray}\ttfamily,
	ndkeywordstyle=\color{editorGreen}\bfseries,
	stringstyle=\color{editorOcher},
	% Code
	language=HTML5,
	alsolanguage=JavaScript,
	alsodigit={.:;},
	tabsize=2,
	showtabs=false,
	showspaces=false,
	showstringspaces=false,
	extendedchars=true,
	breaklines=true,        
	% Support for German umlauts
	literate=%
	{Ö}{{\"O}}1
	{Ä}{{\"A}}1
	{Ü}{{\"U}}1
	{ß}{{\ss}}1
	{ü}{{\"u}}1
	{ä}{{\"a}}1
	{ö}{{\"o}}1
}




























% macros form components generation paper

% !TEX root =  paper.tex

\usepackage{booktabs}
\usepackage[caption=false,font=normalsize,labelfont=sf,textfont=sf]{subfig}
%\usepackage{cite}
%\usepackage{hyperref}
\usepackage{array}
\usepackage{threeparttable}
\usepackage{enumitem}    
%\usepackage[greek,english]{babel}
\usepackage{ifthen}
\usepackage{xspace}
\usepackage{fancybox}
\usepackage{marginnote}
\usepackage{tcolorbox}
\usepackage{multirow}
\usepackage{mathtools}
\usepackage{algpseudocode, algorithm, algorithmicx}
\usepackage{color,soul}
\usepackage{graphicx}
%\usepackage{amsmath, amssymb}
\usepackage{tikz}
%\usepackage{cleveref}
\usepackage{hhline}
\usepackage{balance}


%\newboolean{showcomments} 
\setboolean{showcomments}{false}
%\ifthenelse{\boolean{showcomments}}
%{\newcommand{\nb}[2]{
%		\fbox{\bfseries\sffamily\scriptsize#1}%
%		{\sf\footnotesize$\blacktriangleright$\textcolor{blue}{\hl{#2}}$\blacktriangleleft$}
%	}
%}
%{\newcommand{\nb}[2]{}}

\definecolor{circled-color}{gray}{0.15}
%\newcommand*\circled[1]{\tikz[inner sep=.1ex,baseline=-.75ex] \node[circle,draw,color=white,fill=circled-color] {#1};}


\newcommand{\VizElem}{visual element}
\newcommand{\VE}{VE}
\DeclareMathOperator*{\argmax}{argmax}
\DeclareMathOperator*{\argmin}{argmin}

%\usepackage{mathptmx}

%\let\temp\rmdefault
%\usepackage{mathpazo}
%\let\rmdefault\temp

%\newcommand\ali[1]{\nb{Ali}{#1}}
%\newcommand\mohammad[1]{\nb{Mohammad}{#1}}
%\newcommand\davood[1]{\nb{Davood}{#1}}

%\newcommand{\todo}[1]{\textcolor{magenta}{\nb{TODO:}{#1}}}

%\newcommand{\MyParagraph}[1]{\par\smallskip\noindent\textbf{#1}}

%\newcommand{\header}[1]{\par\vspace{-1mm}\medskip\noindent\textbf{#1.}}

\newcommand{\hide}[1]{}

%\newcommand{\code}[1]{\texttt{\fontsize{9.5}{11}\selectfont #1}}
\newcommand{\smcode}[1]{\texttt{\fontsize{7.5}{8}\selectfont #1}}

\definecolor{findingsbox-bg-color}{gray}{0.90}
\newtcbox{\findingsbox}{colback=findingsbox-bg-color, boxrule=0.2pt, arc=2pt, boxsep=0pt, left=5pt, right=5pt, top=5pt, bottom=5pt}
\newcommand{\findings}[2] {
	\vspace{5pt}
	\noindent
	\findingsbox{
		\begin{minipage}{.95\linewidth}
			\textbf{#1}: #2
		\end{minipage}
	}
}

%\newtheorem{defn}{Definition}

% Define tool names, etc. here
%\newcommand{\html}{\textsc{HTML}\xspace}
%\newcommand{\css}{\textsc{CSS}\xspace}
%\newcommand{\javascript}{\textsc{JavaScript}\xspace}
\newcommand{\react}{\textsc{React}\xspace}
\newcommand{\angular}{\textsc{Angular}\xspace}
\newcommand{\nicad}{\textsc{NiCad}\xspace}
\newcommand{\xpath}{\textsc{XPath}\xspace}
\newcommand{\dom}{\textsc{DOM}\xspace}
\newcommand{\cssdev}{\textsc{CSSDev}\xspace}

\newcommand{\model}{Component Intermediate Model\xspace}
\newcommand{\mappedset}{Mapping Nodes Set\xspace}

%\newcommand{\toolname}{\textsc{VizMod}\xspace}

%\newcommand{\numberOfEmails}{\hl{X}\xspace}
\newcommand{\numberOfTemplates}{five\xspace}
\newcommand{\numberOfParticipants}{five\xspace}
\newcommand{\numberOfComponents}{25\xspace}
\newcommand{\totalNumberOfComponentInstances}{120\xspace}
\newcommand{\precision}{94\%\xspace}
\newcommand{\recall}{75\%\xspace}
\newcommand{\sizeReductionMin}{6.01\%\xspace}
\newcommand{\sizeReductionmax}{19.34\%\xspace}


\newboolean{showreviewhints} 
\setboolean{showreviewhints}{false}
\ifthenelse{\boolean{showreviewhints}}{
	\definecolor{reviewcolor}{RGB}{0,0,255}
	\newcommand{\revised}[2]{\setlength{\marginparwidth}{1.05cm}\marginnote{\color{reviewcolor}{\fbox{\parbox{\dimexpr\linewidth-1\fboxsep-2\fboxrule}{#2}}}}{\color{reviewcolor}{#1}}}
}
{
%	\newcommand{\revised}[2]{#1}
}












%
%
%% This file provides examples of some useful macros for typesetting
%% dissertations.  None of the macros defined here are necessary beyond
%% for the template documentation, so feel free to change, remove, and add
%% your own definitions.
%%
%% We recommend that you define macros to separate the semantics
%% of the things you write from how they are presented.  For example,
%% you'll see definitions below for a macro \file{}: by using
%% \file{} consistently in the text, we can change how filenames
%% are typeset simply by changing the definition of \file{} in
%% this file.
%% 
%
%\newcommand{\NA}{\textsc{n/a}}	% for "not applicable"
%\newcommand{\eg}{e.g.,\ }	% proper form of examples (\eg a, b, c)
%\newcommand{\ie}{i.e.,\ }	% proper form for that is (\ie a, b, c)
%\newcommand{\etal}{\emph{et al}}
%
%% Some useful macros for typesetting terms.
%\newcommand{\file}[1]{\texttt{#1}}
%\newcommand{\class}[1]{\texttt{#1}}
%\newcommand{\latexpackage}[1]{\href{http://www.ctan.org/macros/latex/contrib/#1}{\texttt{#1}}}
%\newcommand{\latexmiscpackage}[1]{\href{http://www.ctan.org/macros/latex/contrib/misc/#1.sty}{\texttt{#1}}}
%\newcommand{\env}[1]{\texttt{#1}}
%\newcommand{\BibTeX}{Bib\TeX}
%
%% Define a command \doi{} to typeset a digital object identifier (DOI).
%% Note: if the following definition raise an error, then you likely
%% have an ancient version of url.sty.  Either find a more recent version
%% (3.1 or later work fine) and simply copy it into this directory,  or
%% comment out the following two lines and uncomment the third.
%\DeclareUrlCommand\DOI{}
%\newcommand{\doi}[1]{\href{http://dx.doi.org/#1}{\DOI{doi:#1}}}
%%\newcommand{\doi}[1]{\href{http://dx.doi.org/#1}{doi:#1}}
%
%% Useful macro to reference an online document with a hyperlink
%% as well with the URL explicitly listed in a footnote
%% #1: the URL
%% #2: the anchoring text
%\newcommand{\webref}[2]{\href{#1}{#2}\footnote{\url{#1}}}
%
%% epigraph is a nice environment for typesetting quotations
%\makeatletter
%\newenvironment{epigraph}{%
%	\begin{flushright}
%	\begin{minipage}{\columnwidth-0.75in}
%	\begin{flushright}
%	\@ifundefined{singlespacing}{}{\singlespacing}%
%    }{
%	\end{flushright}
%	\end{minipage}
%	\end{flushright}}
%\makeatother
%
%% \FIXME{} is a useful macro for noting things needing to be changed.
%% The following definition will also output a warning to the console
%\newcommand{\FIXME}[1]{\typeout{**FIXME** #1}\textbf{[FIXME: #1]}}

% END
